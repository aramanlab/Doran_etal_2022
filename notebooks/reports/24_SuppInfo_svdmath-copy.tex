% Options for packages loaded elsewhere
\PassOptionsToPackage{unicode}{hyperref}
\PassOptionsToPackage{hyphens}{url}
\PassOptionsToPackage{dvipsnames,svgnames,x11names}{xcolor}
%
\documentclass[
  letterpaper,
  DIV=11,
  numbers=noendperiod]{scrartcl}

\usepackage{amsmath,amssymb}
\usepackage{iftex}
\ifPDFTeX
  \usepackage[T1]{fontenc}
  \usepackage[utf8]{inputenc}
  \usepackage{textcomp} % provide euro and other symbols
\else % if luatex or xetex
  \usepackage{unicode-math}
  \defaultfontfeatures{Scale=MatchLowercase}
  \defaultfontfeatures[\rmfamily]{Ligatures=TeX,Scale=1}
\fi
\usepackage{lmodern}
\ifPDFTeX\else  
    % xetex/luatex font selection
\fi
% Use upquote if available, for straight quotes in verbatim environments
\IfFileExists{upquote.sty}{\usepackage{upquote}}{}
\IfFileExists{microtype.sty}{% use microtype if available
  \usepackage[]{microtype}
  \UseMicrotypeSet[protrusion]{basicmath} % disable protrusion for tt fonts
}{}
\makeatletter
\@ifundefined{KOMAClassName}{% if non-KOMA class
  \IfFileExists{parskip.sty}{%
    \usepackage{parskip}
  }{% else
    \setlength{\parindent}{0pt}
    \setlength{\parskip}{6pt plus 2pt minus 1pt}}
}{% if KOMA class
  \KOMAoptions{parskip=half}}
\makeatother
\usepackage{xcolor}
\usepackage{svg}
\setlength{\emergencystretch}{3em} % prevent overfull lines
\setcounter{secnumdepth}{3}
% Make \paragraph and \subparagraph free-standing
\ifx\paragraph\undefined\else
  \let\oldparagraph\paragraph
  \renewcommand{\paragraph}[1]{\oldparagraph{#1}\mbox{}}
\fi
\ifx\subparagraph\undefined\else
  \let\oldsubparagraph\subparagraph
  \renewcommand{\subparagraph}[1]{\oldsubparagraph{#1}\mbox{}}
\fi


\providecommand{\tightlist}{%
  \setlength{\itemsep}{0pt}\setlength{\parskip}{0pt}}\usepackage{longtable,booktabs,array}
\usepackage{calc} % for calculating minipage widths
% Correct order of tables after \paragraph or \subparagraph
\usepackage{etoolbox}
\makeatletter
\patchcmd\longtable{\par}{\if@noskipsec\mbox{}\fi\par}{}{}
\makeatother
% Allow footnotes in longtable head/foot
\IfFileExists{footnotehyper.sty}{\usepackage{footnotehyper}}{\usepackage{footnote}}
\makesavenoteenv{longtable}
\usepackage{graphicx}
\makeatletter
\def\maxwidth{\ifdim\Gin@nat@width>\linewidth\linewidth\else\Gin@nat@width\fi}
\def\maxheight{\ifdim\Gin@nat@height>\textheight\textheight\else\Gin@nat@height\fi}
\makeatother
% Scale images if necessary, so that they will not overflow the page
% margins by default, and it is still possible to overwrite the defaults
% using explicit options in \includegraphics[width, height, ...]{}
\setkeys{Gin}{width=\maxwidth,height=\maxheight,keepaspectratio}
% Set default figure placement to htbp
\makeatletter
\def\fps@figure{htbp}
\makeatother

\KOMAoption{captions}{tableheading}
\makeatletter
\makeatother
\makeatletter
\makeatother
\makeatletter
\@ifpackageloaded{caption}{}{\usepackage{caption}}
\AtBeginDocument{%
\ifdefined\contentsname
  \renewcommand*\contentsname{Table of contents}
\else
  \newcommand\contentsname{Table of contents}
\fi
\ifdefined\listfigurename
  \renewcommand*\listfigurename{List of Figures}
\else
  \newcommand\listfigurename{List of Figures}
\fi
\ifdefined\listtablename
  \renewcommand*\listtablename{List of Tables}
\else
  \newcommand\listtablename{List of Tables}
\fi
\ifdefined\figurename
  \renewcommand*\figurename{Supplement Figure}
\else
  \newcommand\figurename{Supplement Figure}
\fi
\ifdefined\tablename
  \renewcommand*\tablename{Table}
\else
  \newcommand\tablename{Table}
\fi
}
\@ifpackageloaded{float}{}{\usepackage{float}}
\floatstyle{ruled}
\@ifundefined{c@chapter}{\newfloat{codelisting}{h}{lop}}{\newfloat{codelisting}{h}{lop}[chapter]}
\floatname{codelisting}{Listing}
\newcommand*\listoflistings{\listof{codelisting}{List of Listings}}
\makeatother
\makeatletter
\@ifpackageloaded{caption}{}{\usepackage{caption}}
\@ifpackageloaded{subcaption}{}{\usepackage{subcaption}}
\makeatother
\makeatletter
\@ifpackageloaded{tcolorbox}{}{\usepackage[skins,breakable]{tcolorbox}}
\makeatother
\makeatletter
\@ifundefined{shadecolor}{\definecolor{shadecolor}{rgb}{.97, .97, .97}}
\makeatother
\makeatletter
\makeatother
\makeatletter
\makeatother
\ifLuaTeX
  \usepackage{selnolig}  % disable illegal ligatures
\fi
\IfFileExists{bookmark.sty}{\usepackage{bookmark}}{\usepackage{hyperref}}
\IfFileExists{xurl.sty}{\usepackage{xurl}}{} % add URL line breaks if available
\urlstyle{same} % disable monospaced font for URLs
\hypersetup{
  pdftitle={Supplementary Discussion},
  pdfauthor={Benjamin Doran et al.},
  colorlinks=true,
  linkcolor={blue},
  filecolor={Maroon},
  citecolor={Blue},
  urlcolor={Blue},
  pdfcreator={LaTeX via pandoc}}

\title{Supplementary Discussion}
\author{Benjamin Doran et al.}
\date{2023-06-14}

\begin{document}
\maketitle
\ifdefined\Shaded\renewenvironment{Shaded}{\begin{tcolorbox}[sharp corners, enhanced, breakable, frame hidden, borderline west={3pt}{0pt}{shadecolor}, boxrule=0pt, interior hidden]}{\end{tcolorbox}}\fi

\renewcommand*\contentsname{Table of contents}
{
\hypersetup{linkcolor=}
\setcounter{tocdepth}{3}
\tableofcontents
}
\hypertarget{relating-the-statistical-structure-of-co-variation-with-taxonomic-relatedness-in-a-simple-example-of-serial-diversifications}{%
\section{Relating the statistical structure of co-variation with
taxonomic relatedness in a simple example of serial
diversifications}\label{relating-the-statistical-structure-of-co-variation-with-taxonomic-relatedness-in-a-simple-example-of-serial-diversifications}}

\hypertarget{incorporating-an-example-of-convergent-evolution-into-the-statistical-structure-of-co-variation}{%
\section{Incorporating an example of convergent evolution into the
statistical structure of
co-variation}\label{incorporating-an-example-of-convergent-evolution-into-the-statistical-structure-of-co-variation}}

\hypertarget{defining-a-spectral-tree-of-newly-sequenced-strains-considering-the-co-evolutionary-pattern-of-previously-sequenced-bacteria}{%
\section{Defining a Spectral Tree of newly sequenced strains considering
the co-evolutionary pattern of previously sequenced
bacteria}\label{defining-a-spectral-tree-of-newly-sequenced-strains-considering-the-co-evolutionary-pattern-of-previously-sequenced-bacteria}}

\hypertarget{how-similarity-and-dissimilarity-are-encoded-into-spectral-components}{%
\section{How similarity and dissimilarity are encoded into spectral
components}\label{how-similarity-and-dissimilarity-are-encoded-into-spectral-components}}

We sought to understand the mathematics of why spectral factorization
reveals hierarchical scales of relatedness. In answer, we found it is
because the similarity and differences between sub-populations are split
across separate spectral components. Specifically, we show that (i) an
ensemble of systems with two sub-populations will have exactly 2
irreducible spectral components up to the exact point at which those
populations become identical; (ii) the change in magnitude for these
spectral components --- their eigenvalues --- are equal and opposite to
each other as we increase the degree of relatedness between the
sub-populations; and (iii) the major eigenvector encodes the similarity
of the sub-populations and the lesser eigenvector encodes the
sub-population's dissimilarity for all points between the extrema where
the sub-populations are identical or completely independent.

As it is not guaranteed that the reader has a background in the required
mathematics, we will split this section into two parts. Section §4.1
will delve into the necessary detail regarding eigenvalue decomposition,
the determinant, and the characteristic polynomial for readers to
understand the connections between these concepts. Section §4.2 will
detail a specific case of an ensemble of 3 systems, and show how
changing the degree of relatedness between these systems changes
specific aspects of the eigenspectrum.

\hypertarget{linear-algebra-background}{%
\subsection{Linear Algebra Background}\label{linear-algebra-background}}

\hypertarget{what-are-eigenvectors}{%
\subsubsection{What are eigenvectors?}\label{what-are-eigenvectors}}

For readers unfamiliar with linear algebra, some of the early
applications for eigen decomposition in the 1700s were developed to
describe the linear transformations of physical systems (rotations,
shifting, scaling, and shearing of rigid bodies) \footnote{Hawkins, T.
  Cauchy and the spectral theory of matrices. Historia Mathematica 2,
  1--29 (1975).}. Of particular interest in these descriptions are the
principle axes or ``eigenvectors'' of the transformation which are the
only vectors that do not change direction during the transform. The
eigenspectrum describes the complete set of axes ``eigenvectors'', and
is defined as the non-zero solutions to this equation

\begin{align}
C\vec{v} = \lambda\vec{v}
\end{align}

In Equation 1, \(C\) is the linear transformation, represented as a
matrix of real numbers; \(\vec{v}\) is the eigenvector, represented as a
list of real numbers; and \(\lambda\) is the eigenvalue, a real number
that shortens or lengthens the eigen vector.

The eigenvectors and eigenvalues are useful precisely because they are
the only stable descriptors of the transformation and can be used to
consistently describe positions both before and after the transform.

To mathematically solve for these eigenvectors, one common technique is
to first find the eigenvalues which are the solutions to Equation 2, and
then substitute these eigenvalues into Equation 1 and solve for each of
the eigenvectors (\(\vec{v}\)). All applications of spectral
factorization (i.e.~SVD and PCA) use this fundamental equation to define
their spectral components (`eigenvectors').

\begin{align}
\det(C - \lambda I) = 0
\end{align}

In Equation 2, \(C\) is again the linear transformation; \(I\) is the
identity matrix of the same size as \(C\), defined as having \(1\)s
along the diagonal and \(0\) for all other entries; \(det()\) is the
determinant function which describes a change in the degrees of freedom
or number of dimensions after a linear transformation; and \(\lambda\)
is the variable that we are trying to infer (i.e., \(\lambda\) equals
the eigenvalue when Equation 2 is true).

As we will see, Equation 2 can also be expressed as the ``characteristic
polynomial'' of the transform.

\begin{align}
\det(\lambda I - C_{n\times n}) = \lambda^n - c_{n-1}\lambda^{n-1} + c_{n-2}\lambda^{n-2}  - c_{n-3}\lambda^{n-3}... = 0
\end{align}

The characteristic polynomial is a monic alternating sign polynomial of
degree \(n\) where \(n\) is the minimum dimension of \(C\). A monic
polynomial means that the leading coefficient is always equal to \(1\).
And alternating sign means that the sign of each subsequent term
alternates in sign from \(+\) to \(-\) and back. The roots of this
polynomial, (i.e., the values of \(\lambda\) that set it equal to zero)
are the eigenvalues of Equation 2.

\hypertarget{rational-for-using-detc-lambda-i-0}{%
\subsubsection{\texorpdfstring{Rational for using
\(\det(C-\lambda I) = 0\)}{Rational for using \textbackslash det(C-\textbackslash lambda I) = 0}}\label{rational-for-using-detc-lambda-i-0}}

To fully intuit the logic behind for using the equation
\(\det(C-\lambda I) = 0\) to solve for eigenvalues, it will help to
review how matrices in linear algebra describe transformation of space.

If we multiply a vector \(\vec{u}\) by a matrix \(C\) we will get a new
vector that may be pointing in a new direction \(\vec{v}\).
\(C\vec{u} = \vec{v}\) To reiterate, there are special vectors for each
matrix \(C\) that do \emph{not} change direction after the
transformation -- \(C\vec{v} = \lambda\vec{v}\). Instead they only
change in magnitude. These are exactly the eigenvectors of the matrix,
and the magnitude (length of the vector) \(\lambda\) is the eigenvalue.
These eigenvectors and values are particularly important for describing
the transformation and describing how points move through the
transformation because they are the only stable axes during the
transformation.

\begin{figure}

{\centering \includesvg{24_SuppInfo_svdmath copy_files/figure-pdf/fig-vectors-output-1.svg}

}

\caption{\label{fig-vectors}example of linear transformation on regular
vector (left) versus an eigen vector of that transformation (right)}

\end{figure}

To solve for the eigenvectors we can do some algebraic manipulations to
factor out \(\vec{v}\) within Equation 1

\begin{align*}
    C\vec{v} &=  \lambda\vec{v} \\
    C\vec{v} - \lambda\vec{v} &= 0 \\
    C\vec{v} - \lambda I \vec{v} &= 0 \\
    (C - \lambda I )\vec{v} &= 0 \\
\end{align*}

After these manipulations we have the eigenvector \(\vec{v}\) multiplied
by \((C - \lambda I)\) a matrix of our original transform \(C\)
subtracting out the eigenvalue on the diagonal.

Notice that making \(\vec{v}\) equal to the zero vector is a trivial
solution for all matrices. Because this solution is true for all
matrices, eigenvectors are generally defined as only the non-zero
solutions to Equation 1.

So when is \(\vec{v}\) non-zero, and yet multiplying by
\((C - \lambda I)\) results in zero? Our prior manipulations were useful
for answering this question because they let us play a quick thought
experiment. Lets take a moment to assume that \((C - \lambda I)\) is
invertible. Invertible means that there exists some matrix
\((C - \lambda I)^{-1}\) that would completely reverse the
transformation and place every vector back where it started, (i.e., the
transformation matrix is canceled by its inverse transformation). So
\((C - \lambda I)^{-1}(C - \lambda I) = I\) the identity matrix because
for any matrix \(\cancel{A^{-1}A}\vec{v} = I\vec{v} = \vec{v}\).

If we place this inverse into the eigenvector equation from before we
see that if \((C - \lambda I)\) is invertible than \(\vec{v}\)
\textbf{must} equal the zero vector.

\begin{align*}
     (C - \lambda I)\vec{v} &= 0 \\
     (C - \lambda I)^{-1} (C - \lambda I)\vec{v} &= (C - \lambda I)^{-1}0 \\
    I\vec{v} &= 0 \\
    \vec{v} &= 0 \\
\end{align*}

From this thought experiment we see that for \(\vec{v}\) to be non-zero
\((C - \lambda I)\) must be non-invertible. A matrix being
non-invertible essentially means that at least 2 vectors are placed in
the same location after the transformation \(A\vec{u} = A\vec{w}\). In
geometric terms, we can think of this as an reduction in dimensionality;
a line being compressed into a single point, a plane being compressed
into a line or point.

We can then interpret our search for eigenvalues as searching for
intrinsic scales of dimensionality. We can expand a sphere from the
origin and at particular radii we are canceling out dimensional axes,
i.e.~for particular values of \(\lambda\) we are subtracting the full
magnitude of a principal axis in our observed data which flattens all
observations along that axis to zero.

The determinant function is how we measure this collapse of
dimensionality. In the 2 dimensional case,the geometric interpretation
behind the determinant is that it measures the area of the unit square
after a transformation by matrix \(C\). If the transform \(C\)
compresses all the points in the area of the unit square onto a single
line or point, (a) the area of the unit square after the transform
equals zero \(\det(C)=0\), (b) this transform is not invertible. The
transform is not invertible because multiple points are moved onto the
same coordinate by the transform, and so to move every point back to its
original location would require information lost by the transform. The
determinant is used for finding eigenvalues because it is the exact
measure of when a transform is non-invertible (i.e.~has a loss of
dimensionality).

\begin{figure}

{\centering \includesvg{24_SuppInfo_svdmath copy_files/figure-pdf/fig-2d-determinant-output-1.svg}

}

\caption{\label{fig-2d-determinant}example of unit square after
different linear transforms. The last case (right) shows example of the
2d plane being compressed to a single line at which point the
determinant equals zero}

\end{figure}

\hypertarget{the-determinant-defined}{%
\subsubsection{The determinant defined}\label{the-determinant-defined}}

Analytically, the determinant of a 2 x 2 matrix is defined as:

\begin{align*}
\det{\begin{pmatrix} a & b \\ c & d \end{pmatrix}} = \begin{vmatrix} a & b \\ c & d \end{vmatrix} = ad - bc
\end{align*}

And it is defined recursively for larger matrices.

\begin{align*}
\det(C_{n\times n}) = \sum_{j=1}^n \sigma_j C_{1,j} \det(C_{-1, -j})
\end{align*}

where \(\sigma_j\) is defined as \(+1\) if \(j\) is odd and \(-1\) if
\(j\) is even; \(C_{-1, -i}\) is an \(n-1 \times n-1\) matrix made by
removing the first row and \(j\)th column.

For example, here is the first layer of recursion for a 3 x 3 matrix,
which is defined as an alternating sum of 2 x 2 determinants weighted by
each complementary element of the top row.

\begin{align*}
\left|\begin{array}{ccc}
    a & b & c \\
    d & e & f \\
    g & h & i \\
\end{array}\right| \\
= a\left|\begin{array}{cc}e&f\\h&i\end{array}\right| -
    b\left|\begin{array}{cc}d&f\\g&i\end{array}\right| +
    c\left|\begin{array}{cc}d&e\\g&h\end{array}\right|
\end{align*}

Expanded fully, the determinant forms a sum of \(n\) factorial (\(n!\))
products where \(n\) is the number of columns in the matrix.

\begin{align*}
aei - afh - bdi + bfg + cdh - ceg
\end{align*}

\hypertarget{the-relationship-between-determinant-and-characteristic-polynomial}{%
\subsubsection{The relationship between determinant and characteristic
polynomial}\label{the-relationship-between-determinant-and-characteristic-polynomial}}

When all elements of the matrix are known, this sum of products
collapses down to a single number: the result of the determinant. When
there is an unknown variable in the matrix, like with with our
eigenvalue problem in Equation 2, the determinant instead collapses to a
polynomial of the unknown variable. This polynomial is the
characteristic polynomial of the matrix.

As we can show with a 3x3 example:

\begin{align*}
0 &= \det(C - \lambda I) \\
&= \left|\begin{array}{ccc}
    a - \lambda & b & c \\
    d & e - \lambda & f \\
    g & h & i - \lambda \\
\end{array}\right| \\
&= (a - \lambda)\left|\begin{array}{cc}e - \lambda&f\\h&i - \lambda\end{array}\right| -
    b\left|\begin{array}{cc}d&f\\g&i - \lambda\end{array}\right| +
    c\left|\begin{array}{cc}d&e - \lambda\\g&h\end{array}\right|
\end{align*}

When expanded into the alternating sum of \(n!\) terms we see that some
of these terms have more instances of \(\lambda\) than others

\begin{align*}
(a - \lambda)(e - \lambda)(i - \lambda) - (a - \lambda)fh - bd(i - \lambda) + bfg + cdh - c(e - \lambda)g
\end{align*}

We can sort by the number of instances of \(\lambda\)

\begin{align*}
(a - \lambda)(e - \lambda)(i - \lambda) - fh(a - \lambda) - bd(i - \lambda) - cg(e - \lambda) + bfg + cdh
\end{align*}

To continue, let's make this more specific and take it term by term.

For this example we will use the matrix

\begin{align*}
C = \begin{bmatrix}1&1&1\\1&1&1\\1&1&1\end{bmatrix}
\end{align*}

So for this first term we can expand

\begin{align*}
(a - \lambda)(e - \lambda)(i - \lambda) &= (1 - \lambda)(1 - \lambda)(1 - \lambda) \\
&= -\lambda^3 + 3\lambda^2 - 3\lambda + 1
\end{align*}

for the second, third, and fourth terms we get

\begin{align*}
-fh(a - \lambda) &= \\
-bd(i - \lambda) &= \\
-cg(e - \lambda) &= \\
-(1\times 1)(1 - \lambda) &= \lambda - 1
\end{align*}

and our fifth and sixth terms both equal \(+1\).

\begin{align*}
bfg = cdh = 1\cdot 1 \cdot 1 =  1
\end{align*}

Putting them together we get

\begin{align*}
&(-\lambda^3 + 3\lambda^2 - 3\lambda + 1) + (\lambda - 1) + (\lambda - 1) + (\lambda - 1) + 1 + 1  \\
&= -\lambda^3 + 3\lambda^2 - 3\lambda + \lambda + \lambda + \lambda - 1 - 1 - 1 + 1 + 1 + 1 \\
&= -\lambda^3 + 3\lambda^2 - 3\lambda + 3\lambda + 3 - 3 \\
&= -\lambda^3 + 3\lambda^2
\end{align*}

To finish, we multiply by -1 to get the canonical form of the
characteristic polynomial because the characteristic polynomial is
technically defined by \(\det(\lambda I - C )\) rather than
\(\det(C - \lambda I)\)

\begin{align*}
\det(\lambda I - C) = \lambda^3 - 3\lambda^2 \text{ when } C = \begin{bmatrix}1&1&1\\1&1&1\\1&1&1\end{bmatrix}
\end{align*}

What we have described so far is that (i) eigenvectors are the stable
principal axes of linear transformations; (ii) The determinant is used
to solve for eigenvalues (and by proxy eigenvectors) because it measures
when a principal axis (i.e., eigenvector) has been collapsed to zero and
canceled out; (iii) the the eigenvalue problem in Equation 2 can be
equivalently represented by the characteristic polynomial simply by
expanding the determinant of a matrix with an unknown variable.

\hypertarget{the-eigenspectrum-and-singular-value-decomposition}{%
\subsubsection{The eigenspectrum and singular value
decomposition}\label{the-eigenspectrum-and-singular-value-decomposition}}

This section has so far discussed how to find an individual
eigenvector-eigenvalue pair (`spectral component'). But it is important
to remember that the original transform \(C\) is made up of the set of
all spectral components. This concept is important for understanding how
spectral factorization can describe data and physical observations
rather than abstract linear transforms. Specifically, the full set of
spectral components allows us to factor any arbitrary matrix into
separate components of information.

To show this factorization, we start with the transform \(C\) and right
multiply by its collection of eigenvectors \(V\) this has the effect of
scaling each eigenvector by it's associated eigenvalue. contained in
matrix \(\Lambda\) with each \(\lambda_i\) on the diagonal. (this is
exactly because these are the vectors that do not change direction and
are only scaled by the transform.)

\begin{align*}
CV = V\Lambda
\end{align*}

Because the collection of eigenvectors are a set of linearly independent
unitary vectors (each vector has a length of \(1\) and is pointed in a
mutually orthogonal direction -- at right angle -- to all other
eigenvectors), we know that \(V\) has an inverse and that inverse is its
transpose \(VV^{-1} = VV^{t} = V^{t}V = I\)

So

\begin{align*}
C = V\Lambda V^t
\end{align*}

This equation tells us how to recreate our linear transform from its set
of spectral components. More explicitly expanding this matrix
multiplication, the transform \(C\) can be equivalently expressed as the
sum

\begin{align*}
C = V\Lambda V^t = \sum_i \lambda_i v_i v_i^t
\end{align*}

Here \(\lambda_i v_i\) is the \(i\)th scaled eigenvector and \(v_i^t\)
is its transpose. The multiplication \(\lambda_i v_i v_i^t\) tells us is
how each original axis (usually measured traits) in \(v_i^t\) gets
transformed by -- projected onto -- this spectral component
\(lambda_i v_i\). It results in a matrix the same size as the original
transform \(C\) and is called a rank 1 transform because it is composed
of only a single vector. This representation emphasizes that each
spectral component is describing one layer of information about the
original matrix \(C\). Taken together the full set of spectral
components completely recapitulates the original matrix.

Let us now introduce the singular value decomposition (SVD). The SVD is
an extension of eigen decomposition for non-square matrices. And allows
the factorization of any matrix into 3 new matrices; \(U\) the left
singular vectors; \(\Sigma\) a matrix with the singular values along the
diagonal; and \(V^t\) the transposed right singular vectors.

\begin{align*}
M = U\Sigma V^t
\end{align*}

To briefly see the relation to the eigen decomposition we can write

\begin{align*}
MM^t &= U\Sigma V^t V \Sigma U^t \\
MM^t &= U\Sigma \cancel{V^t V} \Sigma U^t \\
MM^t &= U\Sigma^2 U^t \\
\end{align*}

where \(MM^t\) is a matrix times its transpose -- for scaled and
centered matrices this is equivalent to the row-wise covariance matrix;
\(U\) is both the left singular vectors of \(M\) and the eigenvectors of
\(MM^t\); and \(\Sigma\) is a matrix with the singular values on the
diagonal elements and \(\Sigma^2\) are the eigenvalues of \(MM^t\).

\hypertarget{how-relatedness-affects-the-eigenspectrum}{%
\subsection{How relatedness affects the
eigenspectrum}\label{how-relatedness-affects-the-eigenspectrum}}

\hypertarget{creating-an-ensemble-of-systems-with-varying-degrees-of-relatedness}{%
\subsubsection{Creating an ensemble of systems with varying degrees of
relatedness}\label{creating-an-ensemble-of-systems-with-varying-degrees-of-relatedness}}

In this section we will use the concepts relating the eigenspectrum, the
determinant, and the characteristic polynomial to show how the
eigenspectrum changes as we increase the relatedness between systems. We
purposefully use the term ``system'' because it is general and no
property discussed is specifically tied to particular scale or field of
physical science. However, to aid intuition we will pose the initial
cases in terms of evolutionary history and genetics.

A system in this case represents a biological organism that can be
described in terms of genetic traits. For simplicity we can simulate an
organism's genome with a vector of \(1\)'s and \(0\)'s, where a \(1\)
represents the presence of a genetic trait and a \(0\) represents the
absence. For example organism \(a\) can be represented as

\begin{align*}
a = \begin{bmatrix}1&1&1&0&0&0\end{bmatrix}
\end{align*}

where organism \(a\) possesses the first three genetic traits and lacks
the last 3.

Likewise, an ensemble of systems or a population of organisms can be
represented as a binary matrix where rows represent each organism and
each column represents a particular genetic trait. for example

\begin{align*}
\begin{bmatrix}c\\b\\a\end{bmatrix} = \begin{bmatrix}1&1&1&0&0&0\\1&1&1&0&0&0\\1&1&1&0&0&0\end{bmatrix} = M_{similar}
\end{align*}

In this example organisms \(a\), \(b\), and \(c\) are all clones of each
other they share the presence of the first 3 genetic traits and the
absence of the last 3 genetic traits. This is a case of the ensemble or
population being identical or ``similar'' to each other.

At the other extreme we could find an ensemble like this

\begin{align*}
\begin{bmatrix}c\\b\\a\end{bmatrix} = \begin{bmatrix}0&0&0&1&1&1\\1&1&1&0&0&0\\1&1&1&0&0&0\end{bmatrix} =  M_{modular}
\end{align*}

Here the full ensemble is split into two sub-populations with no shared
genetic traits between \(c\) and \(\{a, b\}\). This is a `modular' case
in the sense that these two sub-populations are completely disconnected
in terms of shared information. The descriptors of one sub-population
are completely orthogonal to the other; the two populations tell us
nothing about each other.

The most common case for biological population is somewhere between
these two extremes. We may find some organisms that are quite similar,
along with other organisms that are more distantly related but still
share some genetic traits

\begin{align*}
\begin{bmatrix}c\\b\\a\end{bmatrix} = \begin{bmatrix}0&0&0&1&1&1\\1&1&0&0&0&1\\1&1&0&0&0&1\end{bmatrix} =  M_{related}
\end{align*}

Our questions are how does the eigenspectrum change as we adjust the
degree of relatedness between systems in the ensemble, and where does
the information about that relatedness fall in the eigenspectrum.

To calculate the eigenspectrum of each ensemble we first calculate the
matrix \(MM^t\). The matrix \(MM^t\) is a square similarity matrix
between each pair of organisms, or more generally between each pair of
rows of \(M\). This is the standard first step toward calculating the
SVD and recall from section §4.1.5 it is how we can calculate the
eigenspectrum of rectangular matrices.

We calculate the similarity matrices for each of the degrees of
relatedness.

\begin{align*}
M_{similar} &= \begin{bmatrix}3&3&3\\3&3&3\\3&3&3\\\end{bmatrix} \\
M_{related} &= \begin{bmatrix}3&1&1\\1&3&3\\1&3&3\\\end{bmatrix} \\
M_{modular} &= \begin{bmatrix}3&0&0\\0&3&3\\0&3&3\\\end{bmatrix} \\
\end{align*}

Here, each number in the matrix represents the number of shared genetic
traits between each pair of organisms. And, note how the only numbers
that change are between \(c\) (top) and \(\{a, b\}\) (bottom, middle
respectively). Because of this isolation, we can easily parameterize
this changing degree of relatedness as the variable gamma \(\gamma\),
which we can smoothly vary to understand the changes in the
eigenspectrum.

Specifically, we are first interested in the changes to this equation
with respect to \(\gamma\)

\begin{align*}
\det\left(\begin{bmatrix}
    \langle c|c \rangle-\lambda&\gamma&\gamma\\
    \gamma&\langle b|b \rangle-\lambda&s\\
    \gamma&s&\langle a|a \rangle-\lambda\\
    \end{bmatrix}\right) = 0
\end{align*}

where, \(\langle c|c \rangle\) is the self similarity of organism \(c\)
with itself. In our cases \(c\) shares \(3\) genetic traits with itself
so \(\langle c|c \rangle = 3\). Likewise,
\(\langle a|a\rangle = \langle b|b\rangle = s = 3\). We are using these
variables help track which pairs of organisms contribute to the terms in
the determinant and coefficients of the characteristic polynomial.

\hypertarget{how-do-the-eigenvalues-change-in-response-to-a-changing-degree-of-relatedness}{%
\subsubsection{How do the eigenvalues change in response to a changing
degree of
relatedness?}\label{how-do-the-eigenvalues-change-in-response-to-a-changing-degree-of-relatedness}}

We sought to understand how the magnitudes of the eigenvalues are
dependent on the degree of relatedness between sub-populations of
systems.

To start, we inserted our example's variables into the first layer of
the expanded determinant.

\begin{align*}
(\langle c|c \rangle-\lambda)\left|\begin{array}{cc}(\langle b|b \rangle- \lambda)&s\\ s&(\langle a|a \rangle-\lambda)\end{array}\right| -
    \gamma\left|\begin{array}{cc}\gamma&s\\\gamma&(\langle a|a \rangle-\lambda)\end{array}\right| +
    \gamma\left|\begin{array}{cc}\gamma&(\langle b|b \rangle-\lambda)\\\gamma&s\end{array}\right|
\end{align*}

Note how if \(\gamma=0\), all the terms that compare the similarity of
\(c\) to \(\{a, b\}\) are zeroed out, and the only remaining first term
corresponds specifically to the similarity of \(c\) onto itself and
separately the determinant of \(a\) and \(b\). Also note, how the
\(\gamma\) variable is only in terms with a single \(\lambda\). What
this means is that the only term in the resulting characteristic
polynomial dependent on \(\gamma\) is the first order term of
\(\lambda\).

Expanded and simplified we find that the characteristic polynomial
including \(\gamma\) is

\begin{align*}
\lambda^3 - 9\lambda^2 + (18 - 2\gamma^2)\lambda
\end{align*}

which is indeed only dependent on \(\gamma\) in the first order term of
\(\lambda\).

Substituting in and varying values of \(\gamma\), i.e., varying the
degree of relatedness gives us different instances of the characteristic
polynomial.

\begin{itemize}
\tightlist
\item
  Modular (\(\gamma = 0\)): \(\lambda^3 - 9\lambda^2 + 18\lambda\)
\item
  Related (\(\gamma = 1\)): \(\lambda^3 - 9\lambda^2 + 16\lambda\)
\item
  Related (\(\gamma = 2\)): \(\lambda^3 - 9\lambda^2 + 10\lambda\)
\item
  Similar (\(\gamma = 3\)): \(\lambda^3 - 9\lambda^2\)
\end{itemize}

We note a trend in the first order term of \(\lambda\); as the
sub-populations become more related the first order coefficient in the
polynomial gets closer and closer to zero. Once the sub-populations are
identical the first order coefficient equals zero.

How does this changing coefficient change the roots (`eigenvalues') of
the polynomial?

Plotting the determinant polynomial with respect to \(\lambda\) and more
specifically the roots of the polynomials with respect to \(\gamma\)
shows a pattern. If \(c\) is completely dissimilar to \(a\) and \(b\),
the first order coefficient, \(c_{n-2}\), is greater than zero and there
are two necessarily non-zero eigenvalues: \(λ_1\) and \(λ_2\). Because
\(a\) and \(b\) are completely similar in our example, λ\_3 is always
necessarily zero.

As the similarity of \(c\) to \(a\) and \(b\) increases and \(c_{n-2}\)
goes to zero, there are constraints on how the set of eigenvalues
change. Computing the eigenvalues as a function of similarity of \(c\)
to \(a\) and \(b\) show that the two eigenvalues change in opposite
directions: \(λ_1\) becomes more positive while \(λ_2\) becomes more
negative. This relationship between \(λ_1\) and \(λ_2\) is a natural
mathematical consequence of solving for the roots of a polynomial with
maximum degree three and minimum degree 1.

!{[}

(left) Characteristic polynomial at

3 different degrees of relatedness between organisms.

(right) Roots of the characteristic polynomial while smoothly varying
\(γ\){]}(24\_SuppInfo\_svdmath
copy\_files/figure-pdf/fig-detbylambdaandgamma-output-1.svg)\{\#fig-detbylambdaandgamma\}

We see that if we start with the characteristic polynomial including the
degree of relatedness variable \(\gamma\).

\begin{align*}
\lambda^3 - 9\lambda^2 + (18 - 2\gamma^2)\lambda
\end{align*}

We can immediately factor out a root and power of \(\lambda\).

\begin{align*}
(\lambda - 0)(\lambda^2 - 9\lambda^1 + (18 - 2\gamma^2))
\end{align*}

This is the third root of the eigenvalue problem and it is always equal
to zero because we have set this example with \(a\) and \(b\) as
identical.

Factoring out this root allows us to use the quadratic formula for the
roots on the remaining 2nd order polynomial:

\begin{align*}
\frac{-b \pm \sqrt{b^2 - 4ac}}{2a} \rightarrow \frac{9 \pm \sqrt{81 - 4(18 - 2\gamma^2)}}{2} = \frac{9 \pm \sqrt{8\gamma^2 + 9}}{2}
\end{align*}

Here we see that the remaining two roots are generically expressed using
quadratic formula

\begin{align*}
(\lambda - 0)\left(\lambda - \frac{9 - \sqrt{8\gamma^2 + 9}}{2}\right)\left(\lambda - \frac{9 + \sqrt{8\gamma^2 + 9}}{2}\right)
\end{align*}

Because \(\sqrt{8\gamma^2 + 9}\) will always be smaller than \(9\) while
\(\gamma < (\langle a|a\rangle = 3)\), we can see that as \(\gamma\)
approaches \(3\) the roots approach \(\{0, 0, +9\}\) respectively. And
likewise, as \(\gamma\) approaches \(0\) the roots approach
\(\{0, +3, +6\}\). These results make some intuitive sense when looking
at Supp. Fig. 3 which plots the determinant value with respect to
\(\lambda\). When \(\gamma\) is large it drives the positive first order
term of the polynomial to zero, which means that the negative second
order term dominates for the region between \(\lambda = 0\) to
approximately \(\lambda = 6\) until finally the positive third order
term overtakes and dominates. In contrast when \(\gamma\) is small, the
positive first order is present and can dominate the polynomial for
small values of \(\lambda\) until the higher order terms begin to
dominate the polynomial.

We also note two important facts:

First, the second eigenvalue

\begin{align*}
\left(\lambda - \frac{9 - \sqrt{8\gamma^2 + 9}}{2}\right)
\end{align*}

will only equal zero when \(\gamma\) is exactly equal to
\(3=\langle a|a \rangle =\langle b|b \rangle = s\) -- only when \(c\) is
exactly identical to \(a\) and \(b\). At any point where there is a
difference in similarity and relatedness between these subpopulations
there will be exactly \(2\) eigenvalues, and thus exactly \(2\) spectral
components.

Second, the two eigenvalues (`roots') will change in equal and opposite
directions as the degree of related \(\gamma\) changes. These dynamics
are caused because the only occasion of \(\gamma\) in the quadratic
formula is behind the \(\pm\) sign, we can see that as \(\gamma\)
changes the roots of the characteristic polynomial have to change by the
same amount in both the positive and negative directions. We also note
that the second eigenvalue very quickly drops off, following a power-law
on the order of \(\gamma^2\).

The results in this section support the idea that small
eigenvalues---and by extension the spectral components they correspond
to---are not noise. Up until sub-populations are exactly equal there are
necessarily small spectral components, and because they roots change by
at least the order of \(\gamma^2\) even not vary related sub-populations
can have very spectral components corresponding to eigenvalues with
extremely small magnitude.

\hypertarget{change-of-eigenvector-contributions-in-a-3x3-ensemble-of-systems-as-a-function-of-similarity}{%
\subsubsection{Change of eigenvector contributions in a 3x3 ensemble of
systems as a function of
similarity}\label{change-of-eigenvector-contributions-in-a-3x3-ensemble-of-systems-as-a-function-of-similarity}}

To better understand the relevance of the information being described by
these minor spectral components, we computed the contribution of
organisms \(a\), \(b\), and \(c\) to eigenvectors \(v_1\) and \(v_2\),
defined by \(λ_1\) and \(λ_2\) respectively.

For eigenvector \(v_1\), we found that the contribution of \(a\) and
\(b\) is relatively constant while the contribution of \(c\) rapidly
changes from zero and asymptotically reaches the same constant as \(a\)
and \(b\). In contrast, for eigenvector \(v_2\) we found that the
contribution of \(c\) is relatively constant while the contribution of
\(a\) and \(b\) rapidly changes from zero to asymptotically reach a
constant value away from that of \(h\). These results indicate that the
eigenvector \(v_1\) defines the similarity between \(a\), \(b\), and
\(c\), whereas eigenvector \(v_2\) defines the difference between \(a\),
\(b\), and \(c\). This relationship underlies using the eigenspectrum to
define different scales of relatedness and is necessarily true
independent of the percent variance harbored by each spectral component.

Yet, how is it that varying the similarity of \(c\) to \(a\) and \(b\)
has different effects on the how each system relates to each other
depending on which eigenvector is being considered?

!{[}

Contribution of organisms onto eigenvectors \(v_1\) and \(v_2\) as the

degree of relatedness \(\gamma\) is varied.{]}(24\_SuppInfo\_svdmath
copy\_files/figure-pdf/fig-contributions-output-1.svg)\{\#fig-contributions\}

We sought to understand why changing the similarity of \(c\) to \(a\)
and \(b\) differentially affects the contribution of each system to
eigenvectors \(v_1\) and \(v_2\). Solving the eigenvectors for the set
of eigenvalues defines a system of equations relating the contribution
of each system to eigenvectors \(v_1\) and \(v_2\) and the similarity of
\(c\) to \(a\) and \(b\). The eigenvector equation
\((C-\lambda I)\vec{v} = 0\) expanded into matrix notation is

\begin{align*}
\begin{bmatrix}
    \langle c|c \rangle-\lambda&\gamma&\gamma\\
    \gamma&\langle b|b \rangle-\lambda&s\\
    \gamma&s&\langle a|a \rangle-\lambda\\
\end{bmatrix}\begin{bmatrix}x_c\\x_b\\x_a\end{bmatrix}= \begin{bmatrix}0\\0\\0\end{bmatrix}
\end{align*}

We are interested in how the contributions \(\{x_a, x_b, x_c\}\) of each
organism change with respect to \(\gamma\). This question necessitates
calculating the partial derivative of the contribution of each system
onto either eigenvector \(v_1\) or \(v_2\) with respect to the
similarity of \(c\) to \(a\) and \(b\). We choose to calculate the
derivative for each organism's contribution starting at the modular
example where the sub-populations are completely independent,
\(\gamma = 0\)

To calculate these derivatives it will be helpful to first calculate the
contribution of each organism at the modular case and for comparison at
the similar case so that we have concrete numbers that we can substitute
into the different variables.

\hypertarget{modular-case-eigenvector-v_1}{%
\paragraph{\texorpdfstring{Modular case: eigenvector
\(v_1\)}{Modular case: eigenvector v\_1}}\label{modular-case-eigenvector-v_1}}

we can start with the modular case's set of equations.

\begin{align*}
\begin{bmatrix}
    3-\lambda&0&0\\
    0&3-\lambda&3\\
    0&3&3-\lambda\
\end{bmatrix}\begin{bmatrix}x_c\\x_b\\x_a\end{bmatrix}= \begin{bmatrix}0\\0\\0\end{bmatrix}
\end{align*}\}

\begin{bmatrix}x_c\\x_b\\x_a\end{bmatrix}

=

\begin{bmatrix}0\\0\\0\end{bmatrix}

\textbackslash end\{align*\}

And then substitute the first root in the modular case \(\lambda = 6\)

\begin{align*}
\begin{bmatrix}
    -3&0&0\\
    0&-3&3\\
    0&3&-3\\
\end{bmatrix}\begin{bmatrix}x_c\\x_b\\x_a\end{bmatrix}= \begin{bmatrix}0\\0\\0\end{bmatrix}
\end{align*}

to solve for \(x_c\) we look to the top row's equation

\begin{align*}
-3x_c + 0x_b + 0x_a = 0
\end{align*}

which is solved by setting \(x_c = 0\)

the equations for \(x_b\) and \(x_a\) work together to show that they
can equal any real number so long as \(x_a = x_b\)

\begin{align*}
3x_a - 3x_b = x_a - x_b = 0
\end{align*}

so to make \([x_c,x_b,x_a]^t\) a unitary vector (i.e., with length equal
to \(1\)), we set \(x_a = x_b = \frac{1}{\sqrt{2}}\)

\hypertarget{modular-case-eigenvector-v_2}{%
\paragraph{\texorpdfstring{Modular case: eigenvector
\(v_2\)}{Modular case: eigenvector v\_2}}\label{modular-case-eigenvector-v_2}}

We can use the same process to find the contribution of each organism
onto the second eigenvector with \(\lambda = 3\)

\begin{align*}
\begin{bmatrix}
    0&0&0\\
    0&0&3\\
    0&3&0\\
\end{bmatrix}\begin{bmatrix}x_c\\x_b\\x_a\end{bmatrix}= \begin{bmatrix}0\\0\\0\end{bmatrix}
\end{align*}

to solve for \(x_c\) we look to the top equation

\begin{align*}
0x_c + 0x_b + 0x_a = 0
\end{align*}

and see that \(x_c\) can be any value.

the equations for \(x_b\) and \(x_a\) work together to show that they
must both equal zero.

\begin{align*}
0x_c + 0x_b + 3x_a = x_a = 0
\end{align*}

\begin{align*}
0x_c + 3x_b + 0x_a = x_b = 0
\end{align*}

so to make \([x_c,x_b,x_a]^t\) a unitary vector (i.e., with length equal
to \(1\)), we set \(x_a = x_b = 0\) and \(x_c = 1\)

\hypertarget{similar-case-eigenvector-v_1}{%
\paragraph{\texorpdfstring{Similar case: eigenvector
\(v_1\)}{Similar case: eigenvector v\_1}}\label{similar-case-eigenvector-v_1}}

We also can look to our similar case and find that all contributions are
uniformly distributed along the first eigenvector, with no other
non-zero eigenvalues or eigenvectors.

\begin{align*}
\begin{bmatrix}
    -6&3&3\\
    3&-6&3\\
    3&3&-6\\
\end{bmatrix}\begin{bmatrix}x_c\\x_b\\x_a\end{bmatrix}= \begin{bmatrix}0\\0\\0\end{bmatrix}
\end{align*}

these are permutations of the same equation

\begin{align*}
x_a + x_b - 2x_c = 0
\end{align*}

\begin{align*}
x_a + x_c - 2x_b = 0
\end{align*}

\begin{align*}
x_c + x_b - 2x_a = 0
\end{align*}

and is solved with \(x_c = x_b = x_a\)

So, to make \([x_c,x_b,x_a]^t\) a unitary vector, we set
\(x_a = x_b = x_c = \frac{1}{\sqrt{3}}\)

\hypertarget{gathering-the-calculations}{%
\paragraph{Gathering the
calculations}\label{gathering-the-calculations}}

After these calculations, we have this table of results showing the
contribution of each organism onto eigenvectors both in the modular case
and in the similar case

\begin{longtable}[]{@{}
  >{\raggedright\arraybackslash}p{(\columnwidth - 8\tabcolsep) * \real{0.2000}}
  >{\raggedright\arraybackslash}p{(\columnwidth - 8\tabcolsep) * \real{0.2000}}
  >{\raggedright\arraybackslash}p{(\columnwidth - 8\tabcolsep) * \real{0.2000}}
  >{\raggedright\arraybackslash}p{(\columnwidth - 8\tabcolsep) * \real{0.2000}}
  >{\raggedright\arraybackslash}p{(\columnwidth - 8\tabcolsep) * \real{0.2000}}@{}}
\toprule\noalign{}
\begin{minipage}[b]{\linewidth}\raggedright
organism contribution
\end{minipage} & \begin{minipage}[b]{\linewidth}\raggedright
modular \(v_1\), \(\lambda=6\)
\end{minipage} & \begin{minipage}[b]{\linewidth}\raggedright
modular \(v_2\), \(\lambda=3\)
\end{minipage} & \begin{minipage}[b]{\linewidth}\raggedright
similar \(v_1\), \(\lambda=9\)
\end{minipage} & \begin{minipage}[b]{\linewidth}\raggedright
similar \(v_2\), \(\lambda=0\)
\end{minipage} \\
\midrule\noalign{}
\endhead
\bottomrule\noalign{}
\endlastfoot
\(x_c\) & \(0\) & \(1\) & \(\frac{1}{\sqrt{3}}\) & \(0\) \\
\(x_b\) & \(\frac{1}{\sqrt{2}}\) & \(0\) & \(\frac{1}{\sqrt{3}}\) &
\(0\) \\
\(x_a\) & \(\frac{1}{\sqrt{2}}\) & \(0\) & \(\frac{1}{\sqrt{3}}\) &
\(0\) \\
\end{longtable}

These are the two extreme cases where the sub-populations are either
completely distinct or completely identical. And what we see is that in
the distinct case the contributions of each sub-population are
partitioned onto separate eigenvalues, such that the larger
sub-population \(\{a, b\}\) is on the larger eigenvector with no
contribution from the smaller sub-population \(c\). Conversely, the
smaller sub-population's contribution is completely on the smaller
eigenvector \(v_2\) with no contribution from the larger sub-population.
Thus, once we have increased the degree of relatedness \(\gamma\) to the
point both sub-populations are identical, the contribution of each
organisms is uniformly weighted on the first eigenvector.

We can now get a sense of what the contributions mean and how different
behavior arises on each eigenvector by exploring how the contributions
change as we move away from the modular case by increasing the degree of
relatedness \(\gamma\). Answering this question entails calculating the
partial derivatives of \(\{x_a, x_b, x_c\}\) with respect to \(\gamma\).
We will perform this calculation across both eigenvectors \(v_1\) and
\(v_2\).

\hypertarget{partial-derivative-fracpartial-x_cpartial-gamma}{%
\paragraph{\texorpdfstring{partial derivative:
\(\frac{\partial x_c}{\partial \gamma}\)}{partial derivative: \textbackslash frac\{\textbackslash partial x\_c\}\{\textbackslash partial \textbackslash gamma\}}}\label{partial-derivative-fracpartial-x_cpartial-gamma}}

Let's start with \(x_c\)

On both \(v_1\) and \(v_2\) we will need to solve this equation.

\begin{align*}
(\langle c|c \rangle-\lambda)x_c + \gamma x_b + \gamma x_a = 0
\end{align*}

we can isolate \(x_c\)

\begin{align*}
(\langle c|c \rangle-\lambda)x_c + \gamma x_b + \gamma x_a &= 0 \\
(\langle c|c \rangle-\lambda)x_c &= -(\gamma x_b + \gamma x_a)\\
x_c &= \frac{-\gamma(x_b + x_a)}{(\langle c|c \rangle-\lambda)}\\
x_c &= \frac{\gamma(x_b + x_a)}{\lambda - \langle c|c \rangle}\\
\end{align*}

Because we are starting from the modular case:

\begin{itemize}
\tightlist
\item
  on \(v_1\) we know that \(x_a = x_b = \frac{1}{\sqrt{2}}\)
\item
  on \(v_2\) we know that \(x_a = x_b = 0\)
\end{itemize}

We then get a different partial derivative for each eigenvector; simply
from the fact the \(x_a\) and \(x_b\) zero out the derivative on
eigenvector \(v_2\) and are non-zero on \(v_1\).

On \(v_1\) we can substitute and simplify

\begin{align*}
\begin{align*}
x_c = \frac{\gamma(x_b + x_a)}{\lambda - \langle c|c \rangle} &= \frac{\gamma(\frac{1}{\sqrt{2}} + \frac{1}{\sqrt{2}})}{\lambda - \langle c|c \rangle} \\
&= \frac{2\gamma}{\sqrt{2}\left(\lambda - \langle c|c \rangle\right)} \\
x_c &= \frac{\sqrt{2}\gamma}{\lambda - \langle c|c \rangle} \\
\end{align*} \textbackslash end\{align*\}

On \(v_2\) we can substitute and simplify

\begin{align*}
x_c = \frac{\gamma(x_b + x_a)}{\lambda - \langle c|c \rangle} &= \frac{\gamma(0 + 0)}{\lambda - \langle c|c \rangle} \\
x_c &= 0
\end{align*}

\hypertarget{partial-derivative-fracpartial-x_bpartial-gamma-fracpartial-x_apartial-gamma}{%
\paragraph{\texorpdfstring{partial derivative:
\(\frac{\partial x_b}{\partial \gamma}\),
\(\frac{\partial x_a}{\partial \gamma}\)}{partial derivative: \textbackslash frac\{\textbackslash partial x\_b\}\{\textbackslash partial \textbackslash gamma\}, \textbackslash frac\{\textbackslash partial x\_a\}\{\textbackslash partial \textbackslash gamma\}}}\label{partial-derivative-fracpartial-x_bpartial-gamma-fracpartial-x_apartial-gamma}}

In contrast, when we we start with \(x_b\)'s equation

\begin{align*}
\gamma x_c + (\langle b|b \rangle-\lambda)x_b + s x_a = 0
\end{align*}

And isolate \(x_b\)

\begin{align*}
\gamma x_c + (\langle c|c \rangle-\lambda)x_b + s x_a &= 0\\
x_b &= \frac{-(\gamma x_c + s x_a)}{(\langle b|b \rangle-\lambda)}\\
x_b &= \frac{(\gamma x_c + s x_a)}{\lambda - \langle b|b \rangle}\\
\end{align*}

We see that on eigenvector \(v_1\), \(x_c=0\) so the partial derivative
is zeroed out. And on eigenvector \(v_2\), because \(x_c=1\) the
derivative with respect to \(\gamma\) is

\begin{align*}
\frac{\partial x_b}{\partial \gamma} = \frac{1}{\lambda - \langle b|b \rangle}
\end{align*}

Using the exact same procedure as for \(x_b\), we find for \(x_a\) that
its partial derivatives are essentially identical

Pulling together all these partial derivatives we start to see a
pattern, and explanation for how contributions of organisms can rise and
fall on different eigenvectors

\begin{longtable}[]{@{}
  >{\raggedright\arraybackslash}p{(\columnwidth - 6\tabcolsep) * \real{0.2500}}
  >{\raggedright\arraybackslash}p{(\columnwidth - 6\tabcolsep) * \real{0.2500}}
  >{\raggedright\arraybackslash}p{(\columnwidth - 6\tabcolsep) * \real{0.2500}}
  >{\raggedright\arraybackslash}p{(\columnwidth - 6\tabcolsep) * \real{0.2500}}@{}}
\toprule\noalign{}
\begin{minipage}[b]{\linewidth}\raggedright
eigenvector
\end{minipage} & \begin{minipage}[b]{\linewidth}\raggedright
partial derivative \(x_c\)
\end{minipage} & \begin{minipage}[b]{\linewidth}\raggedright
partial derivative \(x_b\)
\end{minipage} & \begin{minipage}[b]{\linewidth}\raggedright
partial derivative \(x_a\)
\end{minipage} \\
\midrule\noalign{}
\endhead
\bottomrule\noalign{}
\endlastfoot
\(\text{eigenvector } v_1\) &
\(\frac{\partial x_c}{\partial \gamma} = \frac{\sqrt{2}}{\lambda_1 - \langle c|c \rangle}\)
& \(\frac{\partial x_b}{\partial \gamma} = 0\) &
\(\frac{\partial x_a}{\partial \gamma} = 0\) \\
\(\text{eigenvector } v_2\) &
\(\frac{\partial x_c}{\partial \gamma} = 0\) &
\(\frac{\partial x_b}{\partial \gamma} = \frac{1}{\lambda_2 - \langle b|b \rangle}\)
&
\(\frac{\partial x_a}{\partial \gamma} = \frac{1}{\lambda_2 - \langle a|a \rangle}\) \\
\end{longtable}

We find that in the limit of a small increase in similarity of \(c\) to
\(a\) and \(b\), the change in the contribution of \(c\) to eigenvector
\(v_1\) is a constant while the that of \(a\) and \(b\) is zero. In
contrast, the change in the contribution of \(c\) to eigenvector \(v_2\)
is zero while that of \(a\) and \(b\) is a constant. Because \(λ_1\) and
\(λ_2\) trend in opposite directions (and away from the singularity at
\(\lambda=3\)), the contribution of \(c\) to eigenvector \(v_1\)
smoothly tends towards that of \(a\) and \(b\) thereby defining relative
system similarity, while the contributions of \(a\) and \(b\) to
eigenvector \(v_2\) smoothly tend away from that of \(c\) thereby
defining the relative dissimilarity between systems.



\end{document}
